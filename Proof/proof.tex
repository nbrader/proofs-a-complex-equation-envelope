\documentclass[11pt]{article}

\usepackage{amsmath, amssymb}
\usepackage{xcolor}
\usepackage{tcolorbox}
\usepackage{hyperref}
\tcbset{colback=white}

\newcommand{\EE}{E\overline{E}}

\begin{document}

\section*{Solving \texorpdfstring{$aE\overline{E} + b\overline{E} + c = 0$}{aE E* + bE* + c = 0}}

We want to determine, for which triples $(a,b,c)\in\mathbb{C}^3$, the equation
\[
a\,E\overline{E} \;+\; b\,\overline{E} \;+\; c \;=\; 0
\]
admits a solution $E\in\mathbb{C}$.

% ------------------------------------------------------------
\subsection*{The case \texorpdfstring{$a=0$}{a=0}}

When $a=0$, the equation reduces to
\[
b\,\overline{E} + c = 0.
\]

\begin{tcolorbox}
\[
\text{If } b\neq 0, \qquad
\overline{E} = -\frac{c}{b}
\quad\Longrightarrow\quad
E = -\overline{\frac{c}{b}}.
\]
So a solution always exists, for every $c\in\mathbb{C}$.
\end{tcolorbox}

\smallskip

\begin{tcolorbox}
If $b = 0$, then the equation is simply $c=0$.

\begin{itemize}
    \item If $b=0$ and $c=0$: every $E$ is a solution.
    \item If $b=0$ and $c\neq 0$: no solution exists.
\end{itemize}
\end{tcolorbox}

Thus for $a=0$, a solution exists unless
\[
(b,c) = (0,c\neq 0).
\]

\bigskip

% ------------------------------------------------------------
\subsection*{From now on, assume \texorpdfstring{$a\neq 0$}{a ≠ 0}}

We divide the original equation by $a$ and define
\[
b' = \frac{b}{a},
\qquad
c' = \frac{c}{a},
\]
so the equation becomes
\[
E\overline{E} + b'\,\overline{E} + c' = 0.
\]

\begin{tcolorbox}
Rewrite as
\[
c' = -\,\EE + b'\,\overline{E}.
\]
\end{tcolorbox}

Let us decompose $E$ into magnitude and direction:
\[
E = E_{\text{size}}\,E_{\text{dir}},
\qquad
|E_{\text{dir}}|=1.
\]
Then
\[
c' = -\,E_{\text{size}}^2 \;+\; b' E_{\text{size}}\,\overline{E_{\text{dir}}}.
\]

\smallskip

For fixed $E_{\text{size}}$, varying $E_{\text{dir}}$ traces out a circle in the complex plane.  
Write
\[
b' = b_\Sigma\,e^{i\theta}, \qquad b_\Sigma = |b'|,
\]
and rotate coordinates so that the circle lies on the real axis.  
Then $c' = c_x + i c_y$ satisfies
\[
(c_x + c_\Sigma)^2 + c_y^2 = (b_\Sigma E_{\text{size}})^2,
\]
where $c_\Sigma$ is a real shift.

\bigskip

% ------------------------------------------------------------
\subsection*{Finding the envelope of these circles}

Define
\[
F(c_x,c_y,z)
  = (c_x + c_\Sigma)^2 + c_y^2 - (b_\Sigma z)^2.
\]
The envelope of the family is obtained from
\[
F = 0,
\qquad
\frac{\partial F}{\partial z} = 0.
\]
We have
\[
\frac{\partial F}{\partial z}
  = -2 b_\Sigma^2\, z,
\]
so the envelope occurs where $z = E_{\text{size}} = 0$.

Substituting $z=0$ into $F=0$ gives
\[
(c_x + c_\Sigma)^2 + c_y^2 = 0.
\]

Undoing the coordinate shifts and simplifications yields the envelope curve
\[
c_y^2
  = -\,c_x\!\left(c_x^2 + \frac{b_\Sigma^2}{4}\right),
\qquad
c_x \le 0.
\]

\begin{figure}[h]
\centering
\includegraphics[width=0.8\textwidth]{../a-complex-equation-envelope.png}
\end{figure}

\begin{tcolorbox}
\[
\boxed{
\begin{array}{c}
\text{The set of } c' \text{ for which a solution } E \text{ exists} \\
\text{is the interior of this envelope.}
\end{array}
}
\]
\end{tcolorbox}

Translating back to $c = a c'$ determines all triples $(a,b,c)$ for which a solution exists.

\end{document}
